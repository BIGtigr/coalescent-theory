\documentclass{article}

\usepackage{amsmath}
\usepackage{color}
\usepackage{listings}
\usepackage{xfrac}

\definecolor{dkgreen}{rgb}{0,0.6,0}
\definecolor{gray}{rgb}{0.5,0.5,0.5}
\definecolor{mauve}{rgb}{0.58,0,0.82}

\lstset{ %
  language=Python,                % the language of the code
  basicstyle=\footnotesize,           % the size of the fonts that are used for the code
  numbers=left,                   % where to put the line-numbers
  numberstyle=\tiny\color{gray},  % the style that is used for the line-numbers
  stepnumber=2,                   % the step between two line-numbers. If it's 1, each line 
                                  % will be numbered
  numbersep=5pt,                  % how far the line-numbers are from the code
  backgroundcolor=\color{white},      % choose the background color. You must add \usepackage{color}
  showspaces=false,               % show spaces adding particular underscores
  showstringspaces=false,         % underline spaces within strings
  showtabs=false,                 % show tabs within strings adding particular underscores
  frame=single,                   % adds a frame around the code
  rulecolor=\color{black},        % if not set, the frame-color may be changed on line-breaks within not-black text (e.g. commens (green here))
  tabsize=2,                      % sets default tabsize to 2 spaces
  captionpos=b,                   % sets the caption-position to bottom
  breaklines=true,                % sets automatic line breaking
  breakatwhitespace=true,        % sets if automatic breaks should only happen at whitespace
  title=\lstname,                   % show the filename of files included with \lstinputlisting;
                                  % also try caption instead of title
  keywordstyle=\color{blue},          % keyword style
  commentstyle=\color{dkgreen},       % comment style
  stringstyle=\color{mauve},         % string literal style
  escapeinside={\%*}{*)},            % if you want to add a comment within your code
  morekeywords={*,...}               % if you want to add more keywords to the set
}

\title{Problem Set 1}
\author{Alec Story \\ \small{avs38}}

\begin{document}
\maketitle

\section{}
    \subsection{}
    $\xi_1 = 2$, $ \xi_2 = 1$, $ \xi_3 = 0$, $ \xi_4 = 1$
    \subsection{}
    \begin{align*}
    \eta_1 =& \frac{\xi_1 + \xi_4}{1 + \delta_{1, 4}} = \frac{2 + 1}{1} = 3 \\
    \eta_2 =& \frac{\xi_2 + \xi_3}{1 + \delta_{2, 3}} = \frac{1 + 0}{1} = 1
    \end{align*}
    \subsection{}
    \begin{align*}
    S =& \sum_{i=1}^{\lfloor \sfrac{n}{2} \rfloor} \eta_i = 4 \\
    \pi =& \frac{1}{\binom{n}{2}}
           \sum_{i=1}^{\lfloor \sfrac{n}{2} \rfloor} i (n - i) \eta_i = 
           \frac{1}{10} \sum_{i=1}^{2} i (n - i) \eta_i = 1.8
    \end{align*}
    \subsection{}
    \pagebreak
\section{}
By assuming infinite population size, we assume that no simultaneous coalescent
events occur.  If we break this assumption, then we could have multiple
coalescent events occuring in the same generation, which could result in
genealogy trees containing nodes of degree greater than two.
\section{}
\subsection{}
    \begin{figure}[h]
    \begin{center}
        % GNUPLOT: LaTeX picture
\setlength{\unitlength}{0.240900pt}
\ifx\plotpoint\undefined\newsavebox{\plotpoint}\fi
\sbox{\plotpoint}{\rule[-0.200pt]{0.400pt}{0.400pt}}%
\begin{picture}(1500,1200)(0,0)
\sbox{\plotpoint}{\rule[-0.200pt]{0.400pt}{0.400pt}}%
\put(151.0,131.0){\rule[-0.200pt]{4.818pt}{0.400pt}}
\put(131,131){\makebox(0,0)[r]{$0$}}
\put(1419.0,131.0){\rule[-0.200pt]{4.818pt}{0.400pt}}
\put(151.0,320.0){\rule[-0.200pt]{4.818pt}{0.400pt}}
\put(131,320){\makebox(0,0)[r]{$0.2$}}
\put(1419.0,320.0){\rule[-0.200pt]{4.818pt}{0.400pt}}
\put(151.0,509.0){\rule[-0.200pt]{4.818pt}{0.400pt}}
\put(131,509){\makebox(0,0)[r]{$0.4$}}
\put(1419.0,509.0){\rule[-0.200pt]{4.818pt}{0.400pt}}
\put(151.0,698.0){\rule[-0.200pt]{4.818pt}{0.400pt}}
\put(131,698){\makebox(0,0)[r]{$0.6$}}
\put(1419.0,698.0){\rule[-0.200pt]{4.818pt}{0.400pt}}
\put(151.0,887.0){\rule[-0.200pt]{4.818pt}{0.400pt}}
\put(131,887){\makebox(0,0)[r]{$0.8$}}
\put(1419.0,887.0){\rule[-0.200pt]{4.818pt}{0.400pt}}
\put(151.0,1076.0){\rule[-0.200pt]{4.818pt}{0.400pt}}
\put(1419.0,1076.0){\rule[-0.200pt]{4.818pt}{0.400pt}}
\put(277.0,131.0){\rule[-0.200pt]{0.400pt}{4.818pt}}
\put(277,90){\makebox(0,0){$50$}}
\put(277.0,1056.0){\rule[-0.200pt]{0.400pt}{4.818pt}}
\put(407.0,131.0){\rule[-0.200pt]{0.400pt}{4.818pt}}
\put(407,90){\makebox(0,0){$100$}}
\put(407.0,1056.0){\rule[-0.200pt]{0.400pt}{4.818pt}}
\put(536.0,131.0){\rule[-0.200pt]{0.400pt}{4.818pt}}
\put(536,90){\makebox(0,0){$150$}}
\put(536.0,1056.0){\rule[-0.200pt]{0.400pt}{4.818pt}}
\put(665.0,131.0){\rule[-0.200pt]{0.400pt}{4.818pt}}
\put(665,90){\makebox(0,0){$200$}}
\put(665.0,1056.0){\rule[-0.200pt]{0.400pt}{4.818pt}}
\put(794.0,131.0){\rule[-0.200pt]{0.400pt}{4.818pt}}
\put(794,90){\makebox(0,0){$250$}}
\put(794.0,1056.0){\rule[-0.200pt]{0.400pt}{4.818pt}}
\put(923.0,131.0){\rule[-0.200pt]{0.400pt}{4.818pt}}
\put(923,90){\makebox(0,0){$300$}}
\put(923.0,1056.0){\rule[-0.200pt]{0.400pt}{4.818pt}}
\put(1052.0,131.0){\rule[-0.200pt]{0.400pt}{4.818pt}}
\put(1052,90){\makebox(0,0){$350$}}
\put(1052.0,1056.0){\rule[-0.200pt]{0.400pt}{4.818pt}}
\put(1181.0,131.0){\rule[-0.200pt]{0.400pt}{4.818pt}}
\put(1181,90){\makebox(0,0){$400$}}
\put(1181.0,1056.0){\rule[-0.200pt]{0.400pt}{4.818pt}}
\put(1310.0,131.0){\rule[-0.200pt]{0.400pt}{4.818pt}}
\put(1310,90){\makebox(0,0){$450$}}
\put(1310.0,1056.0){\rule[-0.200pt]{0.400pt}{4.818pt}}
\put(1439.0,131.0){\rule[-0.200pt]{0.400pt}{4.818pt}}
\put(1439,90){\makebox(0,0){$500$}}
\put(1439.0,1056.0){\rule[-0.200pt]{0.400pt}{4.818pt}}
\put(151.0,131.0){\rule[-0.200pt]{0.400pt}{227.650pt}}
\put(151.0,131.0){\rule[-0.200pt]{310.279pt}{0.400pt}}
\put(1439.0,131.0){\rule[-0.200pt]{0.400pt}{227.650pt}}
\put(151.0,1076.0){\rule[-0.200pt]{310.279pt}{0.400pt}}
\put(30,603){\makebox(0,0){P(same MRCA)}}
\put(795,29){\makebox(0,0){j}}
\put(151,131){\usebox{\plotpoint}}
\multiput(151.58,131.00)(0.493,26.903){23}{\rule{0.119pt}{20.992pt}}
\multiput(150.17,131.00)(13.000,635.429){2}{\rule{0.400pt}{10.496pt}}
\multiput(164.58,810.00)(0.493,4.461){23}{\rule{0.119pt}{3.577pt}}
\multiput(163.17,810.00)(13.000,105.576){2}{\rule{0.400pt}{1.788pt}}
\multiput(177.58,923.00)(0.493,1.805){23}{\rule{0.119pt}{1.515pt}}
\multiput(176.17,923.00)(13.000,42.855){2}{\rule{0.400pt}{0.758pt}}
\multiput(190.58,969.00)(0.493,0.972){23}{\rule{0.119pt}{0.869pt}}
\multiput(189.17,969.00)(13.000,23.196){2}{\rule{0.400pt}{0.435pt}}
\multiput(203.58,994.00)(0.493,0.616){23}{\rule{0.119pt}{0.592pt}}
\multiput(202.17,994.00)(13.000,14.771){2}{\rule{0.400pt}{0.296pt}}
\multiput(216.00,1010.58)(0.590,0.492){19}{\rule{0.573pt}{0.118pt}}
\multiput(216.00,1009.17)(11.811,11.000){2}{\rule{0.286pt}{0.400pt}}
\multiput(229.00,1021.59)(0.824,0.488){13}{\rule{0.750pt}{0.117pt}}
\multiput(229.00,1020.17)(11.443,8.000){2}{\rule{0.375pt}{0.400pt}}
\multiput(242.00,1029.59)(1.123,0.482){9}{\rule{0.967pt}{0.116pt}}
\multiput(242.00,1028.17)(10.994,6.000){2}{\rule{0.483pt}{0.400pt}}
\multiput(255.00,1035.59)(1.378,0.477){7}{\rule{1.140pt}{0.115pt}}
\multiput(255.00,1034.17)(10.634,5.000){2}{\rule{0.570pt}{0.400pt}}
\multiput(268.00,1040.60)(1.797,0.468){5}{\rule{1.400pt}{0.113pt}}
\multiput(268.00,1039.17)(10.094,4.000){2}{\rule{0.700pt}{0.400pt}}
\multiput(281.00,1044.61)(2.695,0.447){3}{\rule{1.833pt}{0.108pt}}
\multiput(281.00,1043.17)(9.195,3.000){2}{\rule{0.917pt}{0.400pt}}
\put(294,1047.17){\rule{2.700pt}{0.400pt}}
\multiput(294.00,1046.17)(7.396,2.000){2}{\rule{1.350pt}{0.400pt}}
\multiput(307.00,1049.61)(2.695,0.447){3}{\rule{1.833pt}{0.108pt}}
\multiput(307.00,1048.17)(9.195,3.000){2}{\rule{0.917pt}{0.400pt}}
\put(320,1052.17){\rule{2.700pt}{0.400pt}}
\multiput(320.00,1051.17)(7.396,2.000){2}{\rule{1.350pt}{0.400pt}}
\put(333,1053.67){\rule{3.132pt}{0.400pt}}
\multiput(333.00,1053.17)(6.500,1.000){2}{\rule{1.566pt}{0.400pt}}
\put(346,1055.17){\rule{2.700pt}{0.400pt}}
\multiput(346.00,1054.17)(7.396,2.000){2}{\rule{1.350pt}{0.400pt}}
\put(359,1056.67){\rule{3.132pt}{0.400pt}}
\multiput(359.00,1056.17)(6.500,1.000){2}{\rule{1.566pt}{0.400pt}}
\put(372,1057.67){\rule{3.132pt}{0.400pt}}
\multiput(372.00,1057.17)(6.500,1.000){2}{\rule{1.566pt}{0.400pt}}
\put(385,1058.67){\rule{3.132pt}{0.400pt}}
\multiput(385.00,1058.17)(6.500,1.000){2}{\rule{1.566pt}{0.400pt}}
\put(398,1059.67){\rule{3.132pt}{0.400pt}}
\multiput(398.00,1059.17)(6.500,1.000){2}{\rule{1.566pt}{0.400pt}}
\put(411,1060.67){\rule{3.132pt}{0.400pt}}
\multiput(411.00,1060.17)(6.500,1.000){2}{\rule{1.566pt}{0.400pt}}
\put(424,1061.67){\rule{3.132pt}{0.400pt}}
\multiput(424.00,1061.17)(6.500,1.000){2}{\rule{1.566pt}{0.400pt}}
\put(437,1062.67){\rule{3.132pt}{0.400pt}}
\multiput(437.00,1062.17)(6.500,1.000){2}{\rule{1.566pt}{0.400pt}}
\put(463,1063.67){\rule{3.132pt}{0.400pt}}
\multiput(463.00,1063.17)(6.500,1.000){2}{\rule{1.566pt}{0.400pt}}
\put(476,1064.67){\rule{3.132pt}{0.400pt}}
\multiput(476.00,1064.17)(6.500,1.000){2}{\rule{1.566pt}{0.400pt}}
\put(450.0,1064.0){\rule[-0.200pt]{3.132pt}{0.400pt}}
\put(502,1065.67){\rule{3.132pt}{0.400pt}}
\multiput(502.00,1065.17)(6.500,1.000){2}{\rule{1.566pt}{0.400pt}}
\put(489.0,1066.0){\rule[-0.200pt]{3.132pt}{0.400pt}}
\put(541,1066.67){\rule{3.132pt}{0.400pt}}
\multiput(541.00,1066.17)(6.500,1.000){2}{\rule{1.566pt}{0.400pt}}
\put(515.0,1067.0){\rule[-0.200pt]{6.263pt}{0.400pt}}
\put(567,1067.67){\rule{3.132pt}{0.400pt}}
\multiput(567.00,1067.17)(6.500,1.000){2}{\rule{1.566pt}{0.400pt}}
\put(554.0,1068.0){\rule[-0.200pt]{3.132pt}{0.400pt}}
\put(619,1068.67){\rule{3.132pt}{0.400pt}}
\multiput(619.00,1068.17)(6.500,1.000){2}{\rule{1.566pt}{0.400pt}}
\put(580.0,1069.0){\rule[-0.200pt]{9.395pt}{0.400pt}}
\put(671,1069.67){\rule{3.132pt}{0.400pt}}
\multiput(671.00,1069.17)(6.500,1.000){2}{\rule{1.566pt}{0.400pt}}
\put(632.0,1070.0){\rule[-0.200pt]{9.395pt}{0.400pt}}
\put(736,1070.67){\rule{3.132pt}{0.400pt}}
\multiput(736.00,1070.17)(6.500,1.000){2}{\rule{1.566pt}{0.400pt}}
\put(684.0,1071.0){\rule[-0.200pt]{12.527pt}{0.400pt}}
\put(815,1071.67){\rule{3.132pt}{0.400pt}}
\multiput(815.00,1071.17)(6.500,1.000){2}{\rule{1.566pt}{0.400pt}}
\put(749.0,1072.0){\rule[-0.200pt]{15.899pt}{0.400pt}}
\put(919,1072.67){\rule{3.132pt}{0.400pt}}
\multiput(919.00,1072.17)(6.500,1.000){2}{\rule{1.566pt}{0.400pt}}
\put(828.0,1073.0){\rule[-0.200pt]{21.922pt}{0.400pt}}
\put(1062,1073.67){\rule{3.132pt}{0.400pt}}
\multiput(1062.00,1073.17)(6.500,1.000){2}{\rule{1.566pt}{0.400pt}}
\put(932.0,1074.0){\rule[-0.200pt]{31.317pt}{0.400pt}}
\put(1283,1074.67){\rule{3.132pt}{0.400pt}}
\multiput(1283.00,1074.17)(6.500,1.000){2}{\rule{1.566pt}{0.400pt}}
\put(1075.0,1075.0){\rule[-0.200pt]{50.107pt}{0.400pt}}
\put(1296.0,1076.0){\rule[-0.200pt]{34.449pt}{0.400pt}}
\put(151.0,131.0){\rule[-0.200pt]{0.400pt}{227.650pt}}
\put(151.0,131.0){\rule[-0.200pt]{310.279pt}{0.400pt}}
\put(1439.0,131.0){\rule[-0.200pt]{0.400pt}{227.650pt}}
\put(151.0,1076.0){\rule[-0.200pt]{310.279pt}{0.400pt}}
\end{picture}

    \end{center}
    \end{figure}
    From the graph, it is clear that the probablity of a sample sharing its MRCA
    with that of the population as a whole rapidly converges to 1 as the sample
    size grows.
\subsection{}
Base case:  j = 1.  $$\frac{j - 1}{j + 1} \cdot \frac{i + 1}{i - 1} =
                      \frac{0}{2} \cdot \frac{i + 1}{i - 1} = 0$$
With only one lineage, it is its own MRCA, and since $i > j$, one lineage cannot
be the MRCA of the entire sample.

Inductive step:
Assume $P(j) = \frac{j - 1}{j + 1} \cdot \frac{i + 1}{i - 1}$.  We show
$P(j+1) = \frac{j}{j + 2} \cdot \frac{i + 1}{i - 1}$ as follows:

I actually couldn't figure out how to finish the proof.  Here's as far as I got:

If we share the MRCA with the larger sample, then it must be the case that at
least two of the individuals in our smaller sample coalesce at the MRCA.  Then,
at least one of the $j+1$ groupings of individuals will coalesce at the MRCA.
Each of these groupings independently coalesces at the MRCA with probablity
$P(j)$, but because together they are not independent, I don't know where to
proceed from herel.

Ultimately, we need to multiply $P(j)$ by $\frac{j + 1}{j - 1} \cdot \frac{j}{j + 2}$ in
order to get $P(j+1)$.

\section{}
The expected length of time during which there are an even number of lineages
present is the sum of $T_i$ for even $i$.  Because all $T_i$ are independent,
the expectation of their sum is the sum of their expectations, so we get
$$\sum_{i=1}^5 E[T_{2i}] = \sum_{i=1}^5 \frac{2}{2i (2i - 1)} = 1.29 $$

The expected time with an odd number of lineages (excluding one lineage, which
would project infinitely backwards in time) is similar:
$$\sum_{i=1}^4 E[T_{2i+1}] = \sum_{i=1}^4 \frac{2}{(2i+1) (2i)} = 0.508 $$
\section{}
When run on a sample size of 10, a $T_{MRCA}$ of 1.287 was obtained, and a
total branch length of 4.139.  When run 20 times on a sample size of 10, for
$T_{MRCA}$, the mean was 1.77, and the standard deviation 1.13, while for
total branch length the mean was 5.38 and the standard deviation 2.59

The expected value for $T_{MRCA}$ is $2 (1 - \sfrac{1}{n}) = 1.8$, and the
standard deviation is 1.076.  The expected value for the total branch length is
5.67 and the standard deviation is 2.48.  In both cases, the experimental data
is not far off.

The running time for 100 individuals is less than one fortieth of a second on a
2.8 GHz machine, over 2000 times faster than the Fisher-Wright model.

\pagebreak

\lstinputlisting[language=Python]{coalescent.py}
\section{}

Because $t$ is usually defined in terms of $N$, we will use $t$ to refer to $t$
scaled to $N$ throughout, and not to $2N$.

\begin{align*}
E(T_{MRCA}) =& E(T_{MRCA} < t_0) + E(T_{MRCA} > t_0) \\
            =& \int_0^{t_0} t \frac{i (i-1)}{2} e^{-\frac{i (i-1)}{2}t} dt +
               \int_{t_0}^{\infty} t\frac{i (i-1)}{2}
                    e^{-\frac{i (i-1)}{2}\frac{t}{2}} dt \\
            =& \int_0^{t_0} t e^{-t} dt +
               \int_{t_0}^{\infty} t e^{-\frac{t}{2}} dt \\
            =& 1-e^{-t_0} (1+t_0)+2 e^{-\frac{t_0}{2}} (2+t_0)
\end{align*}

\end{document}
